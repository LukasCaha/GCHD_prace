\documentclass[12pt,a4paper]{report}
\usepackage[total={16.5cm,25.2cm}, top=2.5cm, left=2.5cm]{geometry}
\usepackage[czech]{babel}
\usepackage[T1]{fontenc}
\usepackage[utf8]{inputenc}

\setlength\parindent{0.5cm} % šířka odsazení prvního řádku odstavce
\linespread{1.25} % řádkování 1.5 dle MS Word


%%% Údaje o práci
% Název práce v jazyce práce (přesně podle zadání)
\def\NazevPrace{Regresní neuronové sítě}
% Jméno autora
\def\AutorPrace{Lukáš Caha}
% Třída autora
\def\TridaAutora{8.M}
% Školní rok
\def\SkolniRok{2017/2018}
% Seminář ve kterém práce vznikla
\def\Seminar{Seminář z programování}
% Datum dokončení práce
\def\DatumDokonceni{\today}


%% Definice různých užitečných maker (viz popis uvnitř souboru)
%%% Tento soubor obsahuje definice různých užitečných maker a prostředí %%%
%%% Další makra připisujte sem, ať nepřekáží v ostatních souborech.     %%%

%%% Užitečné balíčky (jsou součástí běžných distribucí LaTeXu)
\usepackage{graphicx}       % vkládání obrázků
\usepackage{indentfirst}    % zavede odsazení 1. odstavce kapitoly
\usepackage[nottoc]{tocbibind} % zajistí přidání seznamu literatury,obrázků a tabulek do obsahu
\let\openright=\clearpage
\usepackage{hyperref}
\hypersetup{unicode}
\hypersetup{breaklinks=true}

%%% Drobné úpravy stylu

% Tato makra přesvědčují mírně ošklivým trikem LaTeX, aby hlavičky kapitol
% sázel příčetněji a nevynechával nad nimi spoustu místa. Směle ignorujte.
\makeatletter
\def\@makechapterhead#1{
  {\parindent \z@ \raggedright \normalfont
   \Huge\bfseries \thechapter. #1
   \par\nobreak
   \vskip 20\p@
}}
\def\@makeschapterhead#1{
  {\parindent \z@ \raggedright \normalfont
   \Huge\bfseries #1
   \par\nobreak
   \vskip 20\p@
}}
\makeatother

% Toto makro definuje kapitolu, která není očíslovaná, ale je uvedena v obsahu.
\def\chapwithtoc#1{
\chapter*{#1}
\addcontentsline{toc}{chapter}{#1}
}

% Trochu volnější nastavení dělení slov, než je default.
\lefthyphenmin=2
\righthyphenmin=2

% Zapne černé "slimáky" na koncích řádků, které přetekly, abychom si
% jich lépe všimli.
\overfullrule=1mm




\begin{document}

%% Titulní strana a různé povinné informační strany
%%% Titulní strana práce a další povinné informační strany

%%% Titulní strana práce

\pagestyle{empty}
\hypersetup{pageanchor=false}

\begin{center}

{\large\textbf{Gymnázium Christiana Dopplera, Zborovská 45, Praha 5}}

\vspace{70mm}

{\Large ROČNÍKOVÁ PRÁCE}
\\ \vspace{4mm}
{\Huge\bfseries\NazevPrace}

\vfill
\end{center}

\begin{tabular}{ll}
Vypracoval: & \AutorPrace \\
Třída: & \TridaAutora \\
Školní rok: & \SkolniRok \\
Seminář: & \Seminar \\
\end{tabular}
\newpage

%%% Strana s čestným prohlášením k diplomové práci
\openright
\hypersetup{pageanchor=true}
\pagestyle{plain}
\pagenumbering{gobble}
\vglue 0pt plus 1fill

\noindent
Prohlašuji, že jsem svou ročníkovou práci napsal samostatně a výhradně s~použitím citovaných pramenů. Souhlasím s~využíváním práce na Gymnáziu Christiana Dopplera pro studijní účely.
\vspace{10mm}

\noindent V Praze dne \DatumDokonceni
\hfill
\AutorPrace

\vspace{10mm}
\noindent \hfill . . . . . . . . . . . . . . . . . .

\vspace{20mm}
\newpage

\openright
\pagestyle{plain}
\pagenumbering{arabic}
\setcounter{page}{2}


%%% Strana s automaticky generovaným obsahem diplomové práce
\tableofcontents

\chapter{Úvod}
Ve světě se nachází mnoho dat v mnoha podobách a v dnešní době se dostáváme do bodu, kdy nestačíme všechny třídit a využívat na 100 \%.

Neuronové sítě jsou vrcholem lidské práce v oblasti informačních technologií. Mohl bych je přirovnat k lidskému mozku. A důvodem proč je zde zmiňuji je právě jejich všestranost. Sítí můžeme pouštět dva typy dat. Jedak lidmi vyhodnocené, a poté nevyhodnocené u nichž budu chtít výsledek. Neuronové sítě se podle prvního typu dat naučí jaká je souvislost mezi vstupem a výstupem a pak můžou přibližně určit výstupy dat druhého typu. Pokud byla v prvé řadě síť správně designovaná můžeme očekávat výsledky s poměrně velikou přesností a rychlostí zpracování, na jakou jsem zvyklí u počítačů.

Touto prací bych chtěl rozebrat neuronové sítě na úroveň pochopitelnou i pro středoškoláky, kteří by chtěli začít se strojovým učením, což je obor zahrnující moderní způsoby práce s umělou inteligencí.

\chapter{Základní pojmy} % NÁZVY KAPITOL NEJVYŠŠÍ ÚROVNĚ

\section{Neuron}
	\subsection{Jádro}
		\paragraph{Aktivace}
		je hodnota mezi nulou a jedničkou $(a=0.73)$. Tahle hodnota určuje míru zapnutosti neuronu. Více aktivované neurony můžou mít větší vliv na neurony v sítí přímo následující. Aktivace neuronů jsou závislé na datech, takže není možné měnit jejich hodnoty přímo.
		\paragraph{Normalizační funkce}
		upravuje příchozí signály, tak aby následně vytvořená hodnota zapadala do rozmezí aktivací. Přijdou-li do neuronu 4 signály všechny s maximální hodnotou, bude aktivace neuronu velmi blízko 1.
	\subsection{Synapse}
		\paragraph{Synapse} je spojení mezi dvěma neurony. Toto spojení zajiťuje, že aktivace neuronu v síti je závislá na aktivacích předchozích neuronů.
		\paragraph{Váha}
		ovlivňuje spoje mezi neurony. Váhy spojení tvoří dohromady povahu sítě. Z libovolných vstupních dat můžu upravováním síly spojení (synapsí) vyvodit libovolné výstupní data. 
\section{Síť}
	Síť se skládá s několik vrstev, které jsou navzájem propojené.
	\paragraph{Vrstva}
	je několik neuronů, které se navzájem neovlivňují, ale jsou ovlivněny stejnými neurony a zároveň ovlivňují stejé neurony.
	\paragraph{Bias}
	je míra vlivu nezávislého na datech. Tato externí síla se stará o vyrovnání sítí s menším počtem neuronů a tím uspoří výpočetní výkon. V překladu je bias šum, který rozostřuje data, aby se výsledná síť nepřizpůsobila až příliš moc trénovacím datům.
		
\section{Pohyb dat}
	\subsection{Vstup}
		\subsubsection{Typy vstupních dat}
			\paragraph{Trénovací data}
			jsou data u nichž používáme vstupy i výstupy pro vylepšování sítě. Pokud výsledná síť uvidí znovu tato data bude na nich mít mnohem lepší výsledky, jelikož je trénovaná speciálně na tyto data a až jako vedlejší produkt je tréning na data podobná.
			\paragraph{Testovací data}
			je soubor vstupů, u nichž je známý i výsledek. Ten ale nikdy není ukázán síti, slouží totiž pro porovnání výsledku sítě s pravidvým výsledkem. Takto získává uživatel statistiky o kvalitě sítě.
			\paragraph{Produkční data}
			jsou důvod proč síť vůbec programujeme. Tyto data dostává síť během běžného používání a počítá k nim výsledky. Není však možnost určit jak by tyto výsledky měli vyjít, a proto nám už zbývá pouze doufat, že síť funguje jak popisuje teorie.
		\paragraph{Scaling}
		je metoda upravení hodnot z našich vstupních dat, tak aby v síti tato data vystupovala pouze jako aktivace. Dobrým příkladem je vstupní hodnota věk. V našich datech se vyskytuje člověk s maximální věkem 100 a minimálním 0. Odpovídající hodnoty aktivace potom budou $100\rightarrow1.0$ a $0\rightarrow0.0$.
	\subsection{Výstup}
		\paragraph{Back-scaling}
		je forma získání dat zpět z neuronové sítě. Pokud zrovna trénujeme, není nutné data získávat a pak je porovnávat s očekávanými výsledky, lepší způsob je očekávané výsledky převést na jazyk, kterým komunikuje síť. Tímto samozdřejmě myslím použít scaling a převést výsledek na hodnotu mezi nulou a jedničkou.
	
\chapter{Vstup}
	\section{Design vstupní vrstvy}
	\section{Scaling}
	\section{Data}
\chapter{Forward-propagation}
	\section{Typy sítí}
		\subsection{Normální}
		\subsection{Deep learning}	
\chapter{Back-propagation}
	\section{Loss function}
		
\chapter{Praktická práce s PC}
	\section{Jazyk}
	Jako vývojový jazyk jsem si zvolil Python. Konkrétně používám verzi 3.6.3 na operačním systému Windows 7 64-bit. Tento jazyk jsem si zvolil především, protože je v oboru výrazně preferovaný. Tomu odpovídá i množství knihoven, které pro python v oblasti strojového učení vzniklo. 
		\subsection{Vývojové prostředí}
		Pro vývoj používám NotePad++. IDE zvýrazňuje klíčová slova ve většině jazyků a není komplikované.
		\subsection{Knihovny}
		Knihovny usnadňují práci s programem. Můžeme si je představit právě jako knihy z knihovny plné funkcí, které jsou většinou velmi jednoduché, ale i přes to vyžadují čas na vytvoření a organizaci zdrojových souborů. Navíc jsou dost často optimalizované na to co dělají.
			\paragraph{NumPy}
			poskytuje značné množství jednoduchých matematických operací a funkcí, ze kterých je strojové učení složené. Hlavní funkce je pro mne násobení matic, kterou je možné rychle vyhodnocovat forward-propagation.
	\section{Git}
		Pro zálohování a verzování používám Git.

\chapter{Závěr}
Toto je závěr mé ročníkové práce.

%%% Seznam použité literatury
\begin{thebibliography}{99}

\bibitem{birge}
Birge J. R., Wets R. J.-B. (1987): Computing bounds for stochastic programing problems by means of a generalized moment problem. \textit{Mathematics of Operations Research} \textbf{12}, 149-162.
\end{thebibliography}

%%% Prostor pro přílohy práce
\chapwithtoc{Přílohy}

\openright
\end{document}
